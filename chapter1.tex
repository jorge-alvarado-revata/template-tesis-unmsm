% !TEX encoding = UTF-8 Unicode
\chapter{Planteamiento del problema}

\section{Situación problemática}


Sección para explicar el problema principal que aborda la tesis. Es la sección en donde el autor se concentra en identificar el foco del problema que quiere abordar. 


\subsection{Importancia}
La relevancia e importancia que identifica el autor respecto al planteamiento del problema.

\subsection{Novedad}

La contribución, la nueva perspectiva o lo que hace diferente el aporte de la investigación.


\subsection{Viabilidad}

La factibilidad y posibilidad del estudio.

\newpage

\section{Formulación del problema}


La pregunta principal de la investigación es:

\begin{enumerate}

	\item P1.(H1) ¿Pregunta principal del estudio?

\end{enumerate}

Las pregunta principal se puede expresar con variantes de la siguiente manera:

\begin{enumerate}

	\item P2. (H2)¿Pregunta secundaria o derivada de la principal?

	\item P3. (H3)¿Tercera o siguientes preguntas vinculadas a la idea principal?

	%\item P4. (H4) ¿Siguiente pregunta?



\end{enumerate}

El proyecto espera contribuir con nuevos elementos para un mejor de tal area de conocimiento.

\section{Justificación de la Investigación}

Los argumentos que justifican la investigación. 


\section{Objetivos de la Investigación}


\subsection{Objetivo general}

Cual es el objetivo principal de la tesis.

\subsection{Objetivo Específicos}

\begin{enumerate}
	\item Objetivos especificos vinculados con las preguntas de la tesis.
	\item Objetivos especificos vinculados con las preguntas de la tesis.
\end{enumerate}


