\chapter{Marco teórico}

\section{Antecedentes del problema}

\subsection {Sección primera}

Lorem ipsum dolor sit amet, consectetur adipiscing elit. Curabitur orci sapien, eleifend at nisl suscipit, egestas venenatis felis. In ex odio, efficitur sed velit at, sollicitudin mattis magna. Aliquam erat volutpat. Integer et sem nisi. Donec eget lacus rhoncus, iaculis augue in, fringilla est. Maecenas ut libero odio. Pellentesque eu pellentesque nisl. Curabitur elit ante, rhoncus et magna dignissim, lobortis vehicula ante. Ut luctus risus interdum neque lobortis gravida. Vivamus non eros nulla. Pellentesque accumsan, purus at scelerisque semper, eros felis vulputate diam, id tempor neque lectus sed arcu. Pellentesque non justo congue lacus pharetra porttitor. Morbi at lacinia nibh. \citep{GRADE1990} Lorem ipsum dolor sit amet, consectetur adipiscing elit. Curabitur orci sapien, eleifend at nisl suscipit, egestas venenatis felis. In ex odio, efficitur sed velit at, sollicitudin mattis magna. Aliquam erat volutpat. Integer et sem nisi. Donec eget lacus rhoncus, iaculis augue in, fringilla est. Maecenas ut libero odio. Pellentesque eu pellentesque nisl. Curabitur elit ante, rhoncus et magna dignissim, lobortis vehicula ante. Ut luctus risus interdum neque lobortis gravida. Vivamus non eros nulla. Pellentesque accumsan, purus at scelerisque semper, eros felis vulputate diam, id tempor neque lectus sed arcu. Pellentesque non justo congue lacus pharetra porttitor. Morbi at lacinia nibh. \citep{GonzalesdeOlarte1990}. Lorem ipsum dolor sit amet, consectetur adipiscing elit. Curabitur orci sapien, eleifend at nisl suscipit, egestas venenatis felis. In ex odio, efficitur sed velit at, sollicitudin mattis magna. Aliquam erat volutpat. Integer et sem nisi. Donec eget lacus rhoncus, iaculis augue in, fringilla est. Maecenas ut libero odio. Pellentesque eu pellentesque nisl. Curabitur elit ante, rhoncus et magna dignissim, lobortis vehicula ante.


\subsection{Siguiente sección}

Lorem ipsum dolor sit amet, consectetur adipiscing elit. Curabitur orci sapien, eleifend at nisl suscipit, egestas venenatis felis. In ex odio, efficitur sed velit at, sollicitudin mattis magna. Aliquam erat volutpat. Integer et sem nisi. Donec eget lacus rhoncus, iaculis augue in, fringilla est. Maecenas ut libero odio. Pellentesque eu pellentesque nisl. Curabitur elit ante, rhoncus et magna dignissim, lobortis vehicula ante. Ut luctus risus interdum neque lobortis gravida. Vivamus non eros nulla. Pellentesque accumsan, purus at scelerisque semper, eros felis vulputate diam, id tempor neque lectus sed arcu. Pellentesque non justo congue lacus pharetra porttitor. Morbi at lacinia nibh. \citep{GRADE1990} Lorem ipsum dolor sit amet, consectetur adipiscing elit. Curabitur orci sapien, eleifend at nisl suscipit, egestas venenatis felis. In ex odio, efficitur sed velit at, sollicitudin mattis magna. Aliquam erat volutpat. Integer et sem nisi. Donec eget lacus rhoncus, iaculis augue in, fringilla est. Maecenas ut libero odio. Pellentesque eu pellentesque nisl. Curabitur elit ante, rhoncus et magna dignissim, lobortis vehicula ante. Ut luctus risus interdum neque lobortis gravida. Vivamus non eros nulla. Pellentesque accumsan, purus at scelerisque semper, eros felis vulputate diam, id tempor neque lectus sed arcu. Pellentesque non justo congue lacus pharetra porttitor. Morbi at lacinia nibh. \citep{Yamada2014}. Lorem ipsum dolor sit amet, consectetur adipiscing elit. Curabitur orci sapien, eleifend at nisl suscipit, egestas venenatis felis. In ex odio, efficitur sed velit at, sollicitudin mattis magna. Aliquam erat volutpat. Integer et sem nisi. Donec eget lacus rhoncus, iaculis augue in, fringilla est. Maecenas ut libero odio. Pellentesque eu pellentesque nisl. Curabitur elit ante, rhoncus et magna dignissim, lobortis vehicula ante.


\newpage

\section{Bases Teóricas Generales}


\subsection{Sub Seccion}


\subsubsection{Sub de Sub Seccion}


Etiam ac lobortis diam, in vulputate dui. Nam luctus congue varius. Aliquam tempus lacinia tortor sed sollicitudin. Phasellus purus nunc, suscipit sed interdum vitae, porttitor sit amet leo. Maecenas porttitor porta velit, non scelerisque nulla tincidunt id. Duis sagittis augue in justo fermentum eleifend ut eleifend quam. Phasellus eget est blandit, sagittis felis ut, tincidunt ligula. Nulla laoreet rhoncus sapien sollicitudin facilisis. Donec non vestibulum sapien. Nunc pellentesque lacinia sem, sit amet ultricies tortor vulputate eu. Morbi rhoncus sagittis leo eget lacinia. Proin auctor, justo in laoreet rutrum, dui urna cursus risus, non viverra massa lacus vitae mauris. Phasellus hendrerit faucibus eros id laoreet. 


\subsubsection{Sub2 de Sub Seccion}


Etiam ac lobortis diam, in vulputate dui. Nam luctus congue varius. Aliquam tempus lacinia tortor sed sollicitudin. Phasellus purus nunc, suscipit sed interdum vitae, porttitor sit amet leo. Maecenas porttitor porta velit, non scelerisque nulla tincidunt id. Duis sagittis augue in justo fermentum eleifend ut eleifend quam. Phasellus eget est blandit, sagittis felis ut, tincidunt ligula. Nulla laoreet rhoncus sapien sollicitudin facilisis. Donec non vestibulum sapien. Nunc pellentesque lacinia sem, sit amet ultricies tortor vulputate eu. Morbi rhoncus sagittis leo eget lacinia. Proin auctor, justo in laoreet rutrum, dui urna cursus risus, non viverra massa lacus vitae mauris. Phasellus hendrerit faucibus eros id laoreet.  

 Nunc pellentesque lacinia sem, sit amet ultricies tortor vulputate eu. Morbi rhoncus sagittis leo eget lacinia. Proin auctor, justo in laoreet rutrum, dui urna cursus risus, non viverra massa lacus vitae mauris. Phasellus hendrerit faucibus eros id laoreet.  .

\begin{enumerate}[label=(\arabic*)] 
\item texto en lista. 
\item texto en lista. 
\item texto en lista. 
\item texto en lista. 

\end{enumerate}

Nulla laoreet rhoncus sapien sollicitudin facilisis. Donec non vestibulum sapien. Nunc pellentesque lacinia sem, sit amet ultricies tortor vulputate eu. Morbi rhoncus sagittis leo eget lacinia. Proin auctor, justo in laoreet rutrum, dui urna cursus risus, non viverra massa lacus vitae mauris. Phasellus hendrerit faucibus eros id laoreet.  . 

\citet[p.84]{CIP2006} 
\begin{displayquote}
Nulla laoreet rhoncus sapien sollicitudin facilisis. Donec non vestibulum sapien. Nunc pellentesque lacinia sem, sit amet ultricies tortor vulputate eu. Morbi rhoncus sagittis leo eget lacinia. Proin auctor, justo in laoreet rutrum, dui urna cursus risus, non viverra massa lacus vitae mauris. Phasellus hendrerit faucibus eros id laoreet.  	
\end{displayquote}
Nulla laoreet rhoncus sapien sollicitudin facilisis. Donec non vestibulum sapien. Nunc pellentesque lacinia sem, sit amet ultricies tortor vulputate eu. Morbi rhoncus sagittis leo eget lacinia. Proin auctor, justo in laoreet rutrum, dui urna cursus risus, non viverra massa lacus vitae mauris. Phasellus hendrerit faucibus eros id laoreet.   

Nulla laoreet rhoncus sapien sollicitudin facilisis. Donec non vestibulum sapien. Nunc pellentesque lacinia sem, sit amet ultricies tortor vulputate eu. Morbi rhoncus sagittis leo eget lacinia. Proin auctor, justo in laoreet rutrum, dui urna cursus risus, non viverra massa lacus vitae mauris. Phasellus hendrerit faucibus eros id laoreet.   

Ut tempor lacus magna, nec pretium purus vehicula in. Sed finibus arcu ac eros tincidunt iaculis. Donec pulvinar augue nulla, id mattis nisi aliquam vitae. Morbi vitae nunc ac risus finibus ullamcorper et eu dolor. Morbi cursus sapien est, at iaculis mauris condimentum eu. Phasellus finibus consequat ornare. Aliquam rhoncus urna ac libero facilisis pretium. Donec vel congue tellus. Aliquam erat volutpat. Etiam mauris magna, ultricies eget efficitur non, sagittis id metus. Pellentesque id vulputate ex, sit amet consequat orci. Nunc condimentum, tortor vitae posuere gravida, arcu ligula pretium dolor, nec ultricies felis libero sed sem. Nullam urna dui, hendrerit sit amet cursus non, viverra in mauris. Nulla vehicula lobortis nisl nec tristique. Ut laoreet elit tortor, in dictum sem pulvinar. 



%\begin{table}[ht]
\begin{table}[h]
\centering
\caption{Sub-áreas de la informática.}
\begin{tabular}[t]{lccc}
\hline
Área&Teoría&Abstracción&Diseño\\
\hline
Algoritmos y Estructura de datos&-&-&-\\
Lenguajes de programación&-&-&-\\
Arquitectura&-&-&-\\
Sistemas operativos y redes&-&-&-\\
Ingeniería del software&-&-&-\\
Base de datos y recuperación de información&-&-&-\\
Inteligencia Artificial y robótica&-&-&-\\
Computación gráfica&-&-&-\\
Interacción persona-computador&-&-&-\\
Ciencia computacional&-&-&-\\
Informática organizacional&-&-&-\\
\hline
\end{tabular}
\end{table}


Ut tempor lacus magna, nec pretium purus vehicula in. Sed finibus arcu ac eros tincidunt iaculis. Donec pulvinar augue nulla, id mattis nisi aliquam vitae. Morbi vitae nunc ac risus finibus ullamcorper et eu dolor. Morbi cursus sapien est, at iaculis mauris condimentum eu. Phasellus finibus consequat ornare. Aliquam rhoncus urna ac libero facilisis pretium. Donec vel congue tellus. Aliquam erat volutpat. Etiam mauris magna, ultricies eget efficitur non, sagittis id metus. Pellentesque id vulputate ex, sit amet consequat orci. Nunc condimentum, tortor vitae posuere gravida, arcu ligula pretium dolor, nec ultricies felis libero sed sem. Nullam urna dui, hendrerit sit amet cursus non, viverra in mauris. Nulla vehicula lobortis nisl nec tristique. Ut laoreet elit tortor, in dictum sem pulvinar in. 

Veamos algunos ejemplos de sub-campos de conocimiento y paradigmas:

%\begin{table}[ht]
\begin{table}[h]
\centering
\caption{Algoritmos y estructuras de datos.}
\begin{tabular}[t]{lc}
\hline
Sub-Area&Paradigma\\
\hline
Complejidad computacional& Teoría\\
Concurrencia& Teoría\\
Algoritmos probabilisticos& Teoría\\
Reconocimiento de patrones& Teoría\\
Algoritmos de grafos& Teoría\\
Divide-y-conquista&Experimentación\\
Programación dinámica&Experimentación\\
Interpretes de estado finito&Experimentación\\
Algoritmos aleatorios&Experimentación\\
Testeo de Algoritmos&Experimentación\\
Librerías de software&Diseño\\
Protocolos de comunicación&Diseño\\

\hline
\end{tabular}
\end{table}

Veamos ahora un ejemplo con la sub-área correspondiente a lenguajes de programación:


%\begin{table}[ht]
\begin{table}[h]
\centering
\caption{Lenguajes de programación}
\begin{tabular}[t]{lc}
\hline
Sub-Area&Paradigma\\
\hline
Lenguajes formales& Teoría\\
Teoría de autómatas& Teoría\\
Maquinas de Turing& Teoría\\
Reconocimiento de patrones& Teoría\\
Teoría de tipos& Teoría\\
Lógica matemática& Teoría\\
Programación funcional&Experimentación\\
Lenguajes procedurales&Experimentación\\
OO analisis y diseño&Experimentación\\
Paradigmas de programación&Experimentación\\
Compiladores e interpretes&Diseño\\
Entornos de programación&Diseño\\
Hojas de cálculo&Diseño\\
Debugging-Tracing&Diseño\\
\hline
\end{tabular}
\end{table}

Un tercer ejemplo, adaptado de \citep{EncycloCC2003} para mostrar esta relación entre sub-área de conocimiento y paradigma, lo tomaremos para la sub-área de sistemas operativos y redes.

\begin{table}[h]
\centering
\caption{Sistemas Operativos y Redes}
\begin{tabular}[t]{lc}
\hline
Sub-Area&Paradigma\\
\hline
Teoría de concurrencia& Teoría\\
Algoritmos de planificación& Teoría\\
Maquinas de Turing& Teoría\\
Teoría de Colas& Teoría\\
Criptología& Teoría\\
Lógica matemática& Teoría\\
Gestión de almacenamiento&Experimentación\\
Lenguajes procedurales&Experimentación\\
Procedimientos remotos&Experimentación\\
Protocolos y servicios&Experimentación\\
Sistemas de tiempo compartido&Diseño\\
Árbol del sistema de archivos&Diseño\\
Controladores de dispositivos&Diseño\\
Capas de protocolos&Diseño\\
\hline
\end{tabular}
\end{table}

Donec ut nunc gravida, venenatis quam non, consequat urna. Fusce a diam suscipit, finibus dolor id, hendrerit lectus. Mauris vestibulum felis eu tellus dapibus, id euismod lorem ultricies. Etiam congue purus quam, sed auctor ipsum scelerisque sed. Donec nulla enim, euismod eget interdum vel, pharetra ut nulla. Cras finibus magna sed fermentum bibendum. Phasellus ante quam, dictum non justo sed, tincidunt laoreet lorem. Praesent quis odio id nulla pretium suscipit. Donec a faucibus ligula, et dapibus tortor. Suspendisse nisl elit, elementum et consequat a, auctor eu ex. Aliquam et varius sem, id sollicitudin felis. Donec accumsan in mi nec semper. Nulla rutrum maximus pretium. 


\section {Estado del Arte}


\subsection{Sub Seccion}


Etiam ac lobortis diam, in vulputate dui. Nam luctus congue varius. Aliquam tempus lacinia tortor sed sollicitudin. Phasellus purus nunc, suscipit sed interdum vitae, porttitor sit amet leo. Maecenas porttitor porta velit, non scelerisque nulla tincidunt id. Duis sagittis augue in justo fermentum eleifend ut eleifend quam. Phasellus eget est blandit, sagittis felis ut, tincidunt ligula. Nulla laoreet rhoncus sapien sollicitudin facilisis. Donec non vestibulum sapien. Nunc pellentesque lacinia sem, sit amet ultricies tortor vulputate eu. Morbi rhoncus sagittis leo eget lacinia. Proin auctor, justo in laoreet rutrum, dui urna cursus risus, non viverra massa lacus vitae mauris. Phasellus hendrerit faucibus eros id laoreet.


\begin{table}[h]
\centering
\caption{Competencias Generales}
\begin{tabular}[t]{l}
\hline
Competencias\\
\hline
Aplicar fundamentos matemáticos e informáticos\\
Perspectiva crítica y creativa en la identificación y solución de problemas\\
Identificar el papel de algoritmos y estructuras de datos\\
Implementar algoritmos y estructuras de datos\\
Aplicar principios de ingeniería del software\\
Comprender las limitaciones de computación\\
\hline
\end{tabular}
\label{tab:compgen1}
\end{table}

En  \ref{tab:compgen1} se muestra un extracto de las competencias generales de un profesional de informática del experimento propuesto por \citep{Ramos2013}, en el documento indican que se basan en \citep{acm2005r}.


\begin{table}[h]
\centering
\caption{Competencias Específicas para informática}
\begin{tabular}[t]{|p{12cm}|}
\hline
Competencias\\
\hline
Modelar y diseñar sistemas entendiendo implicaciones y alternativas\\
Utilizar teoría, práctica y herramientas para el proceso de un sistema\\
Aplicar principios de gestión eficaz, organización y habilidades de recuperación de información\\
Aplicar principios de HCI y construir elementos de computación gráfica\\
Implementar el uso eficazmente de herramientas del ciclo de desarrollo de software \\
\hline
\end{tabular}
\label{tab:compgen2}
\end{table}


En  \ref{tab:compgen2} se muestra un extracto de las competencias específicas de un profesional de informática. En el documento indican que se basan en las competencias específicas para ``Computer Science"\space, la ultima revisión específica para informática como ciencia se encuentra en \citep{acm2013}.


Al revisar las competencias resumidas, puede identificarse también cierta generalidad y falto de precisión en lo que corresponde a una competencia. Es más al igual que sucede con las diferencias de competencias entre un informático y un ingeniero del software, las competencias parecerán muy similares.

Al final el proyecto es una guía de auto-evaluación, por lo que cada institución al final revisa dos variables por cada curso o componente educativo: La cobertura de la competencia y la intensidad de la competencia. La cobertura de la competencias es un indicador del porcentaje de competencias que son evaluadas parcial o totalmente dentro de una categoría. La intensidad indica el porcentaje de cursos que desarrollan total o parcialmente las competencias de una categoría.


\begin{table}[h]
\centering
\caption{Resultados Intensidad y cobertura de competencias: Ingeniería en Sistemas ORT}
\begin{tabular}[t]{lcc}
\hline
Competencias&Intensidad&Cobertura\\
\hline
Generales&20\%&92\%\\
ciencia informática&17\%&91.7\%\\
sistemas de información&7\%&100\%\\
ingeniería del software&16\%&100\%\\
ingeniería de computadoras&4\%&100\%\\
tecnología de la información&6\%&85.7\%\\
\hline
\end{tabular}
\label{tab:compgen3}
\end{table}


\subsection {Algoritmo de Máximo común sub-grafo}\label{Algoritmo}

El algoritmo de máximo común sub-grafo es un problema NP por lo que su ejecución se complica conforme crece el número de vértices de los grafos a comparar. El problema es un área de estudio propiamente dicho y de mucha aplicación en el mundo de la química informática. Tiene distintos algoritmos e implementaciones desde la década de los 70 hasta la actualidad. En la presente propuesta usaremos una implementación que identifica el máximo común sub-grafo conectado y usaremos ese valor para el calculo de similitud entre los programas de estudio. 

La decisión para elegir el algoritmo que mostraremos se fundamenta en principios pragmáticos que nos permitan alcanzar los resultados experimentales de la solución al problema; no es el propósito de la investigación evaluar el método de implementación del algoritmo de máximo común sub-grafo, considerando que el número de nodos a comparar en nuestros ejemplos es relativamente pequeño por lo que un tiempo de ejecución optimo no es significativo.

A continuación se explica de modo general una modelo de algoritmo que usamos en el presente proyecto. En esta implementación se considera a los sub-grafos conectados pues permite encontrar la estructura de relaciones comunes de mayor alcance. 

\SetKwInput{KwInput}{Input}
\SetKwInput{KwOutput}{Output}

\begin{algorithm}[H]
    \KwInput{$Comun(G,H)$}
	\KwData{$G$ y  $H$ dos nodos.}
    T = nuevoGrafo()\;
    \For{$E(x,y)$ en H}{

        \If {G tiene $E(x,y)$} {
        	T add($E(x,y)$)\;
        }            

    }
    \KwResult {$T$}\; 
    \caption{Grafo con elementos comunes a G y H}
\end{algorithm}


Suspendisse eget faucibus sem, quis mattis lacus. Vivamus et leo molestie, suscipit tellus faucibus, finibus lacus. Phasellus non dolor cursus, pellentesque neque vel, condimentum tellus. Aliquam aliquam massa eget massa vestibulum gravida. Quisque quis arcu et orci porta sollicitudin eu vitae nunc. Aenean sed hendrerit eros. Sed auctor, tellus et suscipit ullamcorper, erat nisi luctus risus, id pulvinar leo massa sed libero. Interdum et malesuada fames ac ante ipsum primis in faucibus.

\label{def:cuantificacion}

\subsection {Cuantificación de relaciones y nodos}

Un dato importante que también puede considerarse en el objetivo de cuantificar la semejanza o distancia entre las comparaciones, es el uso de una operación simple de conjuntos entre los elementos a comparar. Dado un grafo $g$ y $g^\prime$ calcular cuantos nodos y enlaces son comunes sin que un nodo existente en $g$ esté presente en $g^\prime$ o viceversa. Este simple número puede ayudar a visibilizar los elementos en común entre ambos grafos.
