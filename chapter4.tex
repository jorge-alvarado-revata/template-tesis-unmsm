\chapter{Metodología}

\section{Tipo de Diseño de Investigación}
La presente investigación se concibe como una investigación experimental en donde se hará uso de una técnica de modelamiento basada en las propiedades de los grafos para encontrar evidencia de relaciones entre los elementos conceptuales en este caso los datos considerados comparables en planes de estudio, guías internacionales o cursos. Los grafos que representan de modo independiente programas académicos y referentes internacionales, servirán para aplicar mecanismos de operación con grafos y lograr valores cuantificables de manera que sigamos a pesar de las diferencias nominales se pueda identificar relaciones cuantitativas entre los programas de estudio y referentes internacionales.

La investigación se inicio de modo exploratorio considerando a modo conceptual que han existido estudios de modelamiento de los planes de estudio y su representación en grafos, a su vez los proyectos que han abordado la problemática de la región se han centrado más en buscar la cercanía de los programas de estudio a referentes internacionales o a encontrar semejanzas en las competencias versus los referentes internacionales. Pero pocos estudios se han dado en tratar de buscar similitudes entre los programas locales sin importar el nombre, es decir la búsqueda de similitud tal cual se hace en la industria cuando contratan egresados de la diversa oferta educativa. Aunque se asume que existe tal semejanza en la oferta, no existe estudios que encuentren esa similitud como la que se propone en el presente proyecto. La investigación también es experimental y orientada al modelamiento en el sentido que la información sobre planes de estudio se modela como grafos no dirigidos para representar unidades que corresponden a curso o unidades de conocimiento que nos permitirá contar con estructuras que luego podremos comparar. 


\section{Unidad de análisis}


La unidad de análisis de este estudio son planes de estudio o de referentes internacionales de la enseñanza de carreras de pre-grado en computación, informática o sistemas, sean licenciaturas o ingenierías. En especial de programas de la oferta académica peruana y en especial de la variada y diversa forma de desarrollo de las propuestas educativas peruanas. La estructura y diseño del desarrollo de la academia peruana se ha desarrollado en distintas orientaciones, en la investigación exploratoria encontramos conceptos asociados con el desarrollo de la innovación educativa y el surgimiento en el país de las tendencias educativas. En la parte experimental de esta investigación, usaremos la información de los planes para construir una representación en grafo no dirigido de unidades de conocimientos que tenga correspondencia con estos planes de estudio. El plan de estudio o el referente o guía internacional corresponde a la arquitectura de conocimientos que se imparten en la formación. Como hemos identificado la formación moderna de cualquier disciplina educativa tiene dos componentes importantes el conocimiento impartido y las competencias que se desarrollan. En este estudio usaremos a los planes y guías internacionales como elementos de estudio para identificar la arquitectura del conocimiento impartido, el estudio de competencias no se considera como parte de la investigación. 


\section{Población de estudio}

La población de nuestro estudio es cualquier plan de estudios de profesiones vinculadas a la informática que se imparte en la educación universitaria del país. Los usuarios de esa información a menudo la utilizan para tomar decisiones sobre que carrera elegir o que cursos son importantes dentro de la propuesta. Los planes de estudio también cambian en el tiempo y son dinámicos por que reflejan por un lado la interpretación institucional de la propuesta pero por otro lado también son un reflejo del enfoque y de las posibilidades de una institución. Por lo tanto son una fotografía del momento de la institución y pueden servir como unidades de estudio. 

Las personas que puede tener interés en este estudio va desde académicos que evalúan el progreso o desarrollo de los planes de estudio; estudiantes postulantes a las carreras y que se enfrentan a dilemas de decisión sobre que carrera elegir; a estudiantes de las carreras que llevan matriculados para conocer las similitudes o diferencias que existen respecto a su formación y las impartidas por otros centros de enseñanza. Finalmente puede ser de interés para aquellos que estudian el desarrollo y evolución de la educación en informática del país. 


\section{Tamaño de la muestra}

En el caso de la presente investigación, usaremos información de las siguientes universidades ubicadas en ciudades de marcada tendencia de cambios en la oferta educativa:

\begin{table}[h]
\centering
\caption{Matriz de planes de estudio considerados en el estudio}
\begin{tabular}[t]{lll}
\hline
Carrera & Universidad &  Año plan\\
\hline
%Ingeniería informática&Universidad Ricardo Palma&Privada&2015&Lima\\%2
%Ingeniería informática&Universidad Cayetano Heredia&Privada&2016&Lima\\%3
%Ingeniería de Sistemas&Universidad de Lima&Privada&2020&Lima\\%4
%Ingeniería de Sistemas&Universidad Cesar Vallejo&Privada&2019&ND\\%5
%Ingeniería de Sistemas de Información&Universidad Privada de Ciencias&Privada&ND&ND\\%6
Ingeniería informática & Universidad Católica & 2020\\%7
%Ingeniería informática&Universidad Nacional de Piura&Publica&2018&Piura\\%8
Ingeniería de Sistemas&Universidad San Marcos&2009\\% 1
Ingeniería de Sistemas&Universidad San Marcos&2017\\% 1
%Ingeniería de Sistemas&Universidad San Agustin&Pública&2017&Arequipa\\%9
Ingeniería de Sistemas&Universidad de Ingeniería&2018\\%10
%Ciencia de la Computación&Universidad San Agustin&Pública&2017&Arequipa\\%11
%Ciencia de la Computación&Universidad de Ingeniería&Pública&2017&Lima\\%12
%Ingeniería Informática&Universidad Villareal&2019\\%10
\hline
\end{tabular}
\label{tab:tabunis}
\end{table}


\begin{table}[h]
\centering
\caption{Matriz de guías internacionales}
\begin{tabular}[t]{llcc}
\hline
Perfil&Organización&Graduación&Año guía\\
\hline
Informática&ACM&Pre-grado&1968\\%1
%Informática&UNESCO&Pre-grado&2000\\%2
Informática&ACM/IEEE&Pre-grado&2003\\%3
Informática&ACM/IEEE&Pre-grado&2013\\%4
%Ingeniería informática&ANECA&Pre-grado&2004\\%5
\hline
\end{tabular}
\label{tab:tabperfil}
\end{table}


\section{Selección de la muestra}
Las muestras seleccionadas corresponden a un grupo de 4 propuestas de distintas universidades y con distintos nombres en la propuesta educativa y de 4 modelos de planes de estudio ejemplo derivados de los referentes internacionales de distinta época. La idea central es poder modelar estas propuestas para alcanzar una representación de grafos que nos permita obtener información cuantitativa del modelo.

Se ha buscado en la oferta pública de las universidades los documentos que resumen los planes de estudio vigentes a la fecha de la realización de la investigación. Se han descargado los documentos públicos y se almacenan en carpetas independientes por universidad. La selección podría ser aleatoria pero dada la poca disponibilidad de recursos de base de datos estandarizados para los planes se ha optado elegir propuestas de los perfiles comunes en la oferta educativa: Ingeniería informática, ingeniería de sistemas, ciencia de la computación.

De igual manera con respecto a los perfiles internacionales que sirven de referentes para innovar o estandarizar los programas de estudio se van a considerar a los marcos referenciales más reconocidos a nivel internacional.

\section{Técnicas de recolección de datos}

La recolección de datos ha requerido la búsqueda de documentos oficiales publicados en Internet por las instituciones rectoras tanto de las universidades que ofertan programas de estudios como de los referentes internacionales. Las listas se encuentran indicadas en \ref{tab:tabunis} y \ref{tab:tabperfil}. La investigación se realizará analizando los documentos y creará una base de datos de términos con una tabla llave-valor donde podremos uniformizar unidades de conocimientos o cursos para luego tener como unidad de comparación y modelamiento en grafos.

A continuación se muestra el listado de actividades requeridas para la recolección de datos:

\begin{enumerate}
	\item Revisión documental sobre innovación en sistemas sociales
	\item Revisión documental sobre hechos históricos del desarrollo de la educación en informática.
	\item Identificar planes de estudio de universidades que se tomaran como parte de los experimentos.
	\item Identificar guías internacionales que se usaran como parte del experimento.
	\item Identificar las propiedades de la teoría de grafos que sirva para la investigación.

\end{enumerate}

Como se ha indicado previamente la información de estos programas se modelarán en grafos para estudiar propiedades de estos. De modo que permitan responder las preguntas de la investigación. 




